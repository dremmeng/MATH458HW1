% LaTeX Article Template - customizing page format
%
% LaTeX document uses 10-point fonts by default.  To use
% 11-point or 12-point fonts, use \documentclass[11pt]{article}
% or \documentclass[12pt]{article}.
\documentclass{article}

% Set left margin - The default is 1 inch, so the following 
% command sets a 1.25-inch left margin.
\setlength{\oddsidemargin}{0.25in}

% Set width of the text - What is left will be the right margin.
% In this case, right margin is 8.5in - 1.25in - 6in = 1.25in.
\setlength{\textwidth}{6in}

% Set top margin - The default is 1 inch, so the following 
% command sets a 0.75-inch top margin.
\setlength{\topmargin}{-0.25in}

% Set height of the text - What is left will be the bottom margin.
% In this case, bottom margin is 11in - 0.75in - 9.5in = 0.75in
\setlength{\textheight}{8in}
\usepackage{fancyhdr}
\usepackage{float}
\usepackage{mathtools}
\usepackage{amsmath}
\usepackage{amssymb}
\usepackage{graphicx}
\usepackage{float}
\graphicspath{ {./} }
\setlength{\parskip}{5pt} 
\pagestyle{fancyplain}
% Set the beginning of a LaTeX document
\begin{document}

\lhead{Drew Remmenga MATH 458}
\rhead{HW \#1}
%\lhead{Independent Study}
%\rhead{R Lab}

\begin{enumerate}

\item 
The group is not closed. The group contains no identity.
\item
\begin{enumerate}
\item Closure. Let $A,B \in SL(2,\mathbb{R})$ and $A \circ B = C$. Then $det(C)=det(AB) = det(A)det(B)=1$ thus $C \in SL(2,\mathbb{R})$
\item Associativity. Suppose $A,B,C \in SL(2,\mathbb{R})$ then $A \circ (B \circ C) = (A \circ B) \circ C$ and then $det(A)det(BC)=det(AB)det(C) = det(A)det(B)det(C)=1$ so closure holds.
\item Identity. Let the identity be the 2x2 unit matrix ($e$). $det(e)=1$ so $e \in SL(2,\mathbb{R})$.
\item Inverse. Suppose $\exists A^{-1},A \in SL(2,\mathbb{R})$ such that $A \circ A^{-1} = A^{-1} \circ A = e$. Then we have $det(A \circ A^{-1}) = det(A^{-1}) = det(e) =1$. So $A^{-1}\in SL(2,\mathbb{R})$.
\end{enumerate}
\item
\[
    \begin{array}{l|*{4}{l}}
        & e   & a   & b & c  \\
    \hline
    e & e & a & b & c  \\
    a & a & b & c & e  \\
    b & b & c & e & a\\
    c & c & e & a & b  \\
    \end{array} 
\]

\[
    \begin{array}{l|*{4}{l}}
        & e   & a   & b & c  \\
    \hline
    e & e & a & b & c  \\
    a & a & e & c & b  \\
    b & b & c & a & e\\
    c & c & b & e & a  \\
    \end{array} 
\]

\[
    \begin{array}{l|*{4}{l}}
        & e   & a   & b & c  \\
    \hline
    e & e & a & b & c  \\
    a & a & c & e & b  \\
    b & b & e & c & a\\
    c & c & b & a & e  \\
    \end{array} 
\]

\[
    \begin{array}{l|*{4}{l}}
        & e   & a   & b & c  \\
    \hline
    e & e & a & b & c  \\
    a & a & e & c & b  \\
    b & b & c & e & a\\
    c & c & b & a & e  \\
    \end{array} 
\]
\item
If $xy = zx$ implies $y=z$ then $xz =zx$ and $xy = yx$ for $\forall x,y,z \in G$ therefore the group $G$ is abelian.
\item
For n=0 the case is trivial and becomes the identity. Consider n+1 case. Then we have: 
\begin{equation*}
\begin{split}
(a^{-1}ba)^{n+1} &= a^{-1}b^{n+1}a \\
a^{-n-1}b^{n+1}a^{n+1} &= a^{-1}b^{n+1}a \\
a^{-1}b^{n+1}a^{1} &= a^{-1}b^{n+1}a\\
a^{-1}b^{n+1}a &= a^{-1}b^{n+1}a\\
\end{split}
\end{equation*}
Then consider the n-1 case. Then we have:
\begin{equation*}
\begin{split}
(a^{-1}ba)^{n-1} &= a^{-1}b^{n-1}a\\
a^{-n+1}b^{n-1}a^{n-1} &= a^{-1}b^{n-1}a\\
a^{1}b^{n-1}a^{-1} &= a^{-1}b^{n-1}a\\
a^{-1}b^{n+1}a &= a^{-1}b^{n+1}a\\
\end{split}
\end{equation*}

\end{enumerate}




\end{document}